\capitulo{7}{Conclusiones y Líneas de trabajo futuras}

En este apartado vamos a exponer las conclusiones extraídas de la
realización de este proyecto y las posibles líneas futuras de desarrollo para
la continuidad del proyecto.

\section{Conclusiones}
Ahora que el proyecto se ha terminado, se puede decir que:
\begin{itemize}
	\item Los distintos objetivos del proyecto se han cumplido satisfactoriamente, creando unas herramientas funcionales que dan opción para una recogida de datos y su futuro análisis.
	\item Se ha podido hacer uso de una gran cantidad de herramientas y conocimientos que han sido enseñados por la universidad durante estos años en la carrera. Gracias al proyecto han sido asentados.
	\item Otro de los objetivos personales que me había propuesto era la adquisición de técnicas que no hubiera usado nunca. He podido indagar más en la teoría de grafos, así como en la creación de la Web y el uso de APIs como la de Instagram.
	\item Con los problemas presentados, he aprendido a manejarme en terrenos desconocidos, teniendo que aprender a tratar con los imprevistos. Esto ofrece un conocimiento, no tanto teórico, sino más bien práctico que me ayuda a progresar en la carrera profesional, porque problemas e imprevistos hay siempre, y hay que aprender a lidiar con ellos bajo presión.

\end{itemize}

\section{Líneas de trabajo futuras}
Me gustaría aclarar que, aunque este trabajo va a ser presentado como TFG, eso no significa que sea un trabajo cerrado y perfecto. Como aprendimos en la carrera, los productos tienen que estar en continuo desarrollo, y es por esto que doy aquí unas pautas para posibles compañeros que quieran tomar mi proyecto para continuarlo.
Respecto a la parte teórica y de investigación veo 2 posibles líneas de trabajo posibles:

\begin{itemize}
	\item Estudio de los hashtags en un periodo temporal: Buscar patrones en los hashtags o las palabras y emoticonos usados en las publicación hacer una búsqueda de posibles tendencias que existan o similitudes entre los perfiles. Por ejemplo, y encontrar cosas del estilo de: usuarios que van a celebrar el Curpillos al Parral van también a la hoguera de San Juan. 
	\item Dar valores a las aristas: Se puede hacer de distintas maneras
	\begin{itemize}
	    \item Contando las veces que una persona interactúa con las publicaciones de otra persona, ya sea con “me gustas” o con comentarios en la foto.
	    \item Contando las amistades reales entre personas según el número de publicaciones que tengan en las cuales aparezcan etiquetadas juntas en la misma foto.
    \end{itemize}
\end{itemize}
Respecto al aspecto y practicidad del proyecto veo:
\begin{itemize}
	    \item -	Aplicación Web: Aunque se ha realizado una web para el análisis de los datos y su muestra por pantalla, falta incluir pasos como el de recogida de datos por su duración, y su ensamblaje con el resto de las partes del proyecto.
    \end{itemize}