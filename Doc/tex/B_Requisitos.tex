\apendice{Especificación de Requisitos}

\section{Introducción}
Este anexo va a describir y formalizar los requisitos que debe de cumplir nuestra herramienta software en cuanto a funcionalidad se respecta.
\section{Objetivos generales}
Los objetivos generales del producto son:
\begin{itemize}
    \item Obtención de la red
    \item Estudio de los usuarios
    \item Implementación de una web para mostrar resultados
\end{itemize}



\section{Catalogo de requisitos}
Los requisitos se dividen en dos tipos:
\subsection{Requisitos Funcionales:}
\begin{itemize}
    \item RF1. Crear perfil en Instagram
    \item RF2. Usar el perfil para loguearse
    \item RF3. Crear red de usuarios a estudiar
    \item RF4. Guardar datos en la Base de Datos
    \item RF5. Realizar división por comunidades
    \item RF6. Búsqueda de Usuarios relevantes
\end{itemize}

\subsection{Requisitos no Funcionales:}
\begin{itemize}
    \item RNF1. Modularidad: El software puede ser continuado por otro alumno, por lo cual debe de ser modular.
    \item RNF2. Visibilidad: La web que enseña los datos debe de mostrar bien los datos, de manera intuitiva y explicativa
\end{itemize}


\subsection{Catálogo de requisitos:}

\subsubsection{Caso de uso 1:}
\begin{itemize}
\item Crear perfil en Instagram
\item Descripción: El usuario debe de crear un perfil en Instagram
\item Requisitos: Ninguno
\item Precondiciones: Tener un correo electrónico y conexión a internet
\item Secuencia normal 	Paso	 Acción
\begin{itemize}
    \item 1	Entrar en la página de Instagram 
    \item 2	Darse de alta con un correo electrónico
\end{itemize}
\item Postcondiciones: Se obtiene una manera de continuar en Instagram haciendo uso de su API
\item Excepciones: Tener un usuario previamente
\item Importancia: Alta
\item Urgencia: Alta
\item Comentarios:Ninguno
\end{itemize}

\subsubsection{Caso de uso 2:}
\begin{itemize}
\item Usar perfil de IG para loguearse
\item Descripción: El usuario debe insertar el usuario y contraseña en los scripts para poder hacer uso del proyecto
\item Requisitos: RF1
\item Precondiciones: Ninguna
\item Secuencia normal 	Paso	 Acción
\begin{itemize}
    \item 1	Entrar en los scripts 
    \item 2	Escribir el usuario y contraseña
\end{itemize}
\item Postcondiciones: Se puede hacer uso de los distintos procesos implementados
\item Excepciones: Ninguna
\item Importancia: Alta
\item Urgencia: Alta
\item Comentarios:Ninguno
\end{itemize}

\subsubsection{Caso de uso 3:}
\begin{itemize}
\item Creación de la red de usuarios
\item Descripción: El usuario debe insertar los valores que van a buscarse en las descripciones para buscar personas que lo cumplan
\item Requisitos: RF2
\item Precondiciones: Tener agregados previamente a 5 personas que vayan a estar en la red para tener un punto para comenzar la búsqueda.
\item Secuencia normal 	Paso	 Acción
\begin{itemize}
    \item 1	Entrar en los scripts 
    \item 2	Escribir la lista de términos que van a buscarse
    \item 3	Correr el Script durante un tiempo largo
\end{itemize}
\item Postcondiciones: Se seguirá a las personas que cumplen los requisitos con nuestra cuenta de Instagram
\item Excepciones: Ninguna
\item Importancia: Alta
\item Urgencia: Alta
\item Comentarios: Puede entrarse en la cuenta de Instagram para ver cuántas personas se han agregado para poder parar el script cuando sean suficientes. Puede durar días este paso.
\end{itemize}


\subsubsection{Caso de uso 4:}
\begin{itemize}
\item Guardar datos en la base de datos
\item Descripción: El usuario debe de correr la aplicación encargada de ver las publicaciones de la gente de la red obteniendo la información.
\item Requisitos: RF2, RF3
\item Precondiciones: Tener instalado Mysql
\item Secuencia normal 	Paso	 Acción
\begin{itemize}
    \item 1	Ejecutar el script
\end{itemize}
\item Postcondiciones: Ninguna
\item Excepciones: Tener un usuario previamente
\item Importancia: Alta
\item Urgencia: Alta
\item Comentarios: El tiempo requerido va a ser proporcional al número de publicaciones de los usuarios y número de usuarios en sí.
\end{itemize}


\subsubsection{Caso de uso 5:}
\begin{itemize}
\item Dividir por comunidades
\item Descripción: El usuario debe de correr la aplicación encargada de marcar las comunidades encontradas en la red.
\item Requisitos: RF4
\item Precondiciones: Tener los datos en CSV
\item Secuencia normal 	Paso	 Acción
\begin{itemize}
    \item 1	Ejecutar el script teniendo los datos en la misma ubicación del programa
\end{itemize}
\item Postcondiciones: Se obtiene una manera de continuar en Instagram haciendo uso de su API
\item Excepciones: Tener un usuario previamente
\item Importancia: Alta
\item Urgencia: Alta
\item Comentarios:Ninguno
\end{itemize}


\subsubsection{Caso de uso 6:}
\begin{itemize}
\item Búsqueda de Usuarios relevantes 
\item Descripción: Se obtendrá una lista con los usuarios más relevantes de la red y comunidades
\item Requisitos: RF2, RF3
\item Precondiciones: Usar Google Colab
\item Secuencia normal 	Paso	 Acción
\begin{itemize}
    \item 1	Subir el notebook a google colab
    \item 2	Ejecutar el Notebook y seguir los pasos que nos va indicando
\end{itemize}
\item Postcondiciones: Ninguna
\item Excepciones: Ninguna
\item Importancia: Alta
\item Urgencia: Alta
\item Comentarios:Ninguno
\end{itemize}


