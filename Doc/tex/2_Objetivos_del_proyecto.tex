\capitulo{2}{Objetivos del proyecto}

El propósito del proyecto es estudiar la red de gente de Burgos y sus relaciones en la red social Instagram.
Además, se pretende desarrollar de una interfaz web para poder hacer análisis de otras redes, no necesariamente de gente de Burgos, sino usando como factor común otro valor como un deporte o una afición, no solo de manera geográfica.

\section{Objetivos globales}

\begin{itemize}
	\item Automatizar el proceso de extracción de los datos de las relaciones de Instagram.
	
	\item Introducir la información en una base de datos para almacenarla
	
	\item Automatizar el estudio de la información.
	
	\item Mantener el usuario que ha permitido la extracción de los datos para que un futuro sea po-sible hacer otro estudio en mayor profundidad.
	
	\item Desarrollar una aplicación web que permita:
	\begin{itemize}
	\item La extracción de datos
	\item La impresión por pantalla de los resultados
	\end{itemize}
	
\end{itemize}

\section{Objetivos técnicos}

\begin{itemize}
    \item Montar un servidor para la obtención de los datos (se requiere de unos días)
    \item Desarrollar una aplicación utilizando Python para la extracción de relaciones de usuarios e in-teracciones entre ellos.
    \item Utilización de Scrum como metodología para la planificación del proyecto, con los sprint definidos en Github.
    \item Utilización de Github como servidor web donde subir las versiones.
    \item Utilización de Flask para el desarrollo de la aplicación Web
    \item Utilización de LaTeX como herramienta con la compilar la documentación.
    \item Uso de Networkx como principal herramienta de estudio de redes.
\end{itemize}

\section{Objetivos personales}
\begin{itemize}

	\item Adquirir conocimientos nuevos que no se han profundizado en la universidad.
	
	\item Aportar a la universidad una investigación para un mejor desarrollo del sector de la investigación en la universidad.
	 
	\item Aprender a llevar un proyecto de duración media, con la distribución de cargas de trabajo.
	 
	\item Aprender a desarrollar una aplicación entera, con el Frontend y el Backend.
	 
\end{itemize}
