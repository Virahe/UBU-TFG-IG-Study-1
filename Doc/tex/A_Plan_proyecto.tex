\apendice{Plan de Proyecto Software}



\section{Introducción}


En esta sección vamos a tratar la planificación seguida durante el proyecto. La planificación es un una parte fundamental a la hora de embarcarse en un proyecto, y es ahí cuando se toman se fijan temas como:
\begin{itemize}
    \item Los objetivos
    \item El tiempo
    \item La financiación
\end{itemize}
Por este motivo la planificación se va a dividir en:
\begin{itemize}
    \item Planificación temporal
    \item Estudio de viabilidad
\end{itemize}




\section{Planificación temporal}

Para la planificación temporal, haremos uso de una metodología ágil: SCRUM.
Se sigue una estrategia con el desarrollo incremental. La duración de los Sprints era de una o dos semanas, dependiendo de las tutorías futuras posibles y factores externos que pudieran entorpecer el ritmo del proyecto.
Se describirán ahora los Sprints realizados:

\subsection{Sprint 0 (6/3/19 – 12/3/19):}

En la primera reunión de planificación del proyecto se hablo de las ideas que iban a llevarse a cabo. Se mandaron las primeras tareas a realizar.
Las tareas que realizar fueron:
\begin{itemize}
    \item Descarga de ciertos Softwares a usar durante 1el proyecto, tales como Balsamic o ZenHub.
    \item Estudio de la políticas y licencias de Instagram.
    \item Investigación del funcionamiento de la API.
\end{itemize}

\subsection{Sprint 1 (13/3/19 – 20/3/19):}

Las tareas a realizar en este Sprint fueron:
\begin{itemize}
    \item Crear una cuenta de Instagram
    \item Conectarme a la API
\end{itemize}


\subsection{Sprint 2 (21/3/19 – 27/3/19):}

Las tareas a realizar en este Sprint fueron:
\begin{itemize}
    \item Investigación sobre como hacer el reconocimiento de la gente de Burgos
    \item Hacer una búsqueda de palabras que se usen en las descripciones de los perfiles.
\end{itemize}


\subsection{Sprint 3 (28/3/19 – 10/4/19):}

Las tareas a realizar en este Sprint fueron:
\begin{itemize}
    \item Espera a los permisos de la conexión de la base de datos
    \item Búsqueda de Bases de Datos
    \item Conexión VPN
\end{itemize}

\subsection{Sprint 5 (11/4/19 – 17/4/19):}

Las tareas a realizar en este Sprint fueron:
\begin{itemize}
    \item Preparar servidor en ordenador personal
    \item Descarga de Screen para Terminal
    \item Instalación de bibliotecas
\end{itemize}

\subsection{Sprint 6 (18/4/19 – 1/5/19):}

Las tareas a realizar en este Sprint fueron:
\begin{itemize}
    \item Instalación del conector de la base de datos
    \item Recopilación de datos
    \item Extracción y maquetación de los datos
\end{itemize}

\subsection{Sprint 7 (2/5/19 – 15/5/19):}

Las tareas a realizar en este Sprint fueron:
\begin{itemize}
    \item Instalación de la biblioteca de Louvain
    \item División por comunidades de los datos
    \item Investigar si hay alguna comunidad clave
\end{itemize}

\subsection{Sprint 8 (9/5/19 – 22/5/19):}

Las tareas a realizar en este Sprint fueron:
\begin{itemize}
    \item Estudio de personas relevantes global
    \item Estudio por comunidades de los usuarios más relevantes
\end{itemize}

\subsection{Sprint 9 (23/5/19 – 5/6/19):}

Las tareas a realizar en este Sprint fueron:
\begin{itemize}
    \item Estudio de Hashtags más usados en toda la red
    \item Estudio de Hashtags más relevantes por comunidades
\end{itemize}

\subsection{Sprint 10 (6/6/19 – 19/6/19):}

Las tareas a realizar en este Sprint fueron:
\begin{itemize}
    \item Aprendizaje de Flask
    \item Realización de la página web donde mostrar los resultados
\end{itemize}

\subsection{Sprint 11 (20/6/19 – 3/7/19):}

Las tareas a realizar en este Sprint fueron:
\begin{itemize}
    \item Pulir detalles del proyecto
    \item Terminar documentación 
    \item Entregar proyecto
\end{itemize}

\subsection{Sprint 12 (4/7/19 - 10/8/19):}

Las tareas a realizar en este Sprint fueron:
\begin{itemize}
    \item Presentar el proyecto
\end{itemize}



\section{Estudio de viabilidad}

Este apartado muestra el estudio realizado sobre la viabilidad económica y legal del producto a entregar.

\subsection{Viabilidad económica}

En la siguiente sección se realiza una aproximación de los costes que tendría que asumir una empresa para llevar adelante este proyecto.

\subsubsection{Coste del personal}
El transcurso del proyecto ha sido de 3 meses desde que el desarrollador junior se puso a trabajar en él. Suponiendo que el desarrollador tiene un salario neto de 1100 euros:

\begin{table}[!h]
	\centering
	\begin{tabular}{@{}l|l@{}}
		\toprule
		Concepto & Coste  \\
		\midrule
		Salario mensual neto & 1100 euros \\
		Seguridad Social (29,9 \%) & 596,91 euros \\
		Salario mensual bruto & 1.696,91 euros \\
		\midrule
		Total 3 meses & 5.090,73 euros \\
		\bottomrule
	\end{tabular}
	\caption{Coste del personal}
	\label{tab:personal}
\end{table}

La retribución a la Seguridad Social es de un 29,9\% siguiendo los siguientes puntos a valorar:


\begin{itemize}
	\item 23,6\% de contingencias comunes.
	\item 5,5\% de desempleo de tipo general.
	\item 0,6\% de formación profesional.
	\item 0,2\% del fondo de garantía salarial.
\end{itemize}

\subsubsection{Coste del material}
En el coste material habría que mirar tanto el coste del Hardware como del Software. En mi caso a nivel de software no ha habido ningún coste puesto que he ido buscando siempre usar software libre. 
Por este motivo solo hay que valorar el coste material del Hardware.
Para la realización del proyecto se ha utilizado el ordenador personal del alumno que tiene un valor de 500 euros al cual se le estima un tiempo para amortizarlo de 6 años. Dado que el tiempo que ha sido de 3 meses:


\begin{table}[!h]
	\centering
	\begin{tabular}{l|l|l}
		\toprule
		Concepto & Coste & Coste amortizado  \\
		\midrule
		Ordenador & 500 euros & 20,83 euros \\
		\bottomrule
	\end{tabular}
	\caption{Coste del Hardware}
	\label{tab:material}
\end{table}

\subsubsection{Coste Total}
La suma de todos los costes es:

\begin{table}[!h]
	\centering
	\begin{tabular}{@{}l|l@{}}
		\toprule
		Concepto & Coste  \\
		\midrule
		Hardware & 20,83 \\
		Personal & 5.090,73 euros \\
		\midrule
		Total & 5.111,56 euros \\
		\bottomrule
	\end{tabular}
	\caption{Coste Total del Proyecto}
	\label{tab:costeTotal}
\end{table}

\subsubsection{Beneficios}

La forma de obtener beneficios es vender la aplicación para realizar estudios de mercado. Basándome en un precio teórico de 250 euros por cada estudio de mercado que se requiera, se tardaría en amortizar el producto 25 estudios de mercado. A partir de ese momento habría beneficios.


\subsection{Viabilidad legal}

En este siguiente apartado se va a estudiar la viabilidad legal del proyecto. Hay que analizar todas las licencias de las bibliotecas que se han usado para ver que licencia se podría aplicar a nuestro producto software.

\begin{table}[!h]
	\centering
	\begin{tabular}{@{}l|l@{}}
		\toprule
		Librería & Licencia  \\
		\midrule
		Python and modules integrated & PSFL[Zotero] \\
		Instagram-API-python & MIT License[Zotero] \\
		Networkx & 3-clause BSD license [Zotero] \\
		\bottomrule
	\end{tabular}
	\caption{Tabla librerías}
	\label{tab:librerias}
\end{table}

Como podemos observar en la tablas, las licencias de las bibliotecas utilizadas son bastante permisivas y tienden a favorecer el open source.
Este proyecto lo publico con la licencia “Apache 2.0”, que dictamina lo siguiente:


\begin{table}[!h]
	\centering
	\begin{tabular}{l|l|l}
		\toprule
		Permissions & Limitations & Conditions  \\
		\midrule
		Commercial use & Trademark & License and copyright notice \\
		Modification & Liability & State changes \\
		Distribution & Warranty \\
		Patent use \\
		Private use \\
		\bottomrule
	\end{tabular}
	\caption{Tabla Licencias}
	\label{tab:licencia}
\end{table}